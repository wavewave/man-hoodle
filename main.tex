%%%%%%%%%%%%%%%%%%%%%%%%%%%%%%%%%%%%%%%%%
% The Legrand Orange Book
% LaTeX Template
% Version 1.1 (11/4/13)
%
% This template has been downloaded from:
% http://www.LaTeXTemplates.com
%
% Original author:
% Mathias Legrand (legrand.mathias@gmail.com)
%
% License:
% CC BY-NC-SA 3.0 (http://creativecommons.org/licenses/by-nc-sa/3.0/)
%
% Compiling this template:
% This template uses biber for its bibliography and makeindex for its index.
% This means that to update the bibliography and index in this template you
% will need to run the following sequence of commands in the template
% directory:
%
% 1) pdflatex main
% 2) makeindex main.idx -s StyleInd.ist
% 3) biber main
% 4) pdflatex main
%
% This template also uses a number of packages which may need to be
% updated to the newest versions for the template to compile. It is strongly
% recommended you update your LaTeX distribution if you have any
% compilation errors.
%
% Important note:
% Chapter heading images should have a 2:1 width:height ratio,
% e.g. 920px width and 460px height.
%
%%%%%%%%%%%%%%%%%%%%%%%%%%%%%%%%%%%%%%%%%

%----------------------------------------------------------------------------------------
%	PACKAGES AND OTHER DOCUMENT CONFIGURATIONS
%----------------------------------------------------------------------------------------

\documentclass[11pt,fleqn]{book} % Default font size and left-justified equations

\usepackage[top=3cm,bottom=3cm,left=3.2cm,right=3.2cm,headsep=10pt,a4paper]{geometry} % Page margins

%\usepackage{hyperref}

\usepackage{xcolor} % Required for specifying colors by name
\definecolor{ocre}{RGB}{243,102,25} % Define the orange color used for highlighting throughout the book

% Font Settings
\usepackage{avant} % Use the Avantgarde font for headings
%\usepackage{times} % Use the Times font for headings
\usepackage{mathptmx} % Use the Adobe Times Roman as the default text font together with math symbols from the Sym­bol, Chancery and Com­puter Modern fonts
\usepackage{microtype} % Slightly tweak font spacing for aesthetics
\usepackage[utf8]{inputenc} % Required for including letters with accents
\usepackage[T1]{fontenc} % Use 8-bit encoding that has 256 glyphs

% Bibliography
\usepackage[style=alphabetic,sorting=nyt,sortcites=true,autopunct=true,babel=hyphen,hyperref=true,abbreviate=false,backref=true,backend=biber]{biblatex}
\addbibresource{bibliography.bib} % BibTeX bibliography file
\defbibheading{bibempty}{}

% Index
\usepackage{calc} % For simpler calculation - used for spacing the index letter headings correctly
\usepackage{makeidx} % Required to make an index
\makeindex % Tells LaTeX to create the files required for indexing

%----------------------------------------------------------------------------------------

\input{structure} % Insert the commands.tex file which contains the majority of the structure behind the template

\begin{document}

\pagenumbering{roman}
\setcounter{page}{1}
%----------------------------------------------------------------------------------------
%	TITLE PAGE
%----------------------------------------------------------------------------------------

\begingroup
\thispagestyle{empty}
\AddToShipoutPicture*{\put(6,5){\includegraphics[scale=1]{background}}} % Image background
\centering
\vspace*{9cm}
\par\normalfont\fontsize{35}{35}\sffamily\selectfont
The Hoodle User's Guide\par % Book title
\vspace*{1cm}
{\Huge version 0.2.0}\par % Author name
\endgroup

%----------------------------------------------------------------------------------------
%	COPYRIGHT PAGE
%----------------------------------------------------------------------------------------

\newpage
~\vfill
\thispagestyle{empty}

\noindent Copyright \copyright\ 2013 Ian-Woo Kim\\ % Copyright notice

\noindent \textsc{Published by ianwookim.org}\\ % Publisher

\noindent \textsc{ianwookim.org}\\ % URL

\noindent Licensed under the Creative Commons Attribution-NonCommercial 3.0 Unported License (the ``License''). You may not use this file except in compliance with the License. You may obtain a copy of the License at \url{http://creativecommons.org/licenses/by-nc/3.0}. Unless required by applicable law or agreed to in writing, software distributed under the License is distributed on an \textsc{``AS IS'' BASIS, WITHOUT WARRANTIES OR CONDITIONS OF ANY KIND}, either express or implied. See the License for the specific language governing permissions and limitations under the License.\\ % License information

\noindent \textit{First printing, April 2013} % Printing/edition date

%----------------------------------------------------------------------------------------
%	TABLE OF CONTENTS
%----------------------------------------------------------------------------------------

\chapterimage{chapter_head_1.pdf} % Table of contents heading image

\pagestyle{fancy} %empty} % No headers

\tableofcontents % Print the table of contents itself

\cleardoublepage % Forces the first chapter to start on an odd page so it's on the right



%%%
%%  Start Part 1
%%

%\frontmatter 
\pagestyle{fancy} % Print headers again


\chapter*{Preface}
\addcontentsline{toc}{chapter}{Preface}

%\section{Development History}

I was a long-time user of \verb|xournal| and contributed some code 
to the \verb|xournal| project myself. The \verb|xournal| program 
is excellent and fulfils my need much for jotting notes,  calculating with a pen 
and annotating papers in pdf format. However, as all the software in the world,
it needs to be improved. For example, we need more optimal effecient use of 
small tablet PC screen.  It is also very desirable to have a script for a program.
On the other hand, \verb|xournal| is already a stable mature program and 
has lots of users. So it already has a big inertia in accepting new interesting 
experimental features that I would like to add. 

This situation got much exacerbated for me since \verb|xournal| is written in C, 
which needs great care for maintaining code safely. I implemented a lasso selection feature 
for \verb|xournal| (it is included as of \verb|xournal-0.4.7|). 
When adding such features, it was clear that some internal 
design need to be significantly changed, but it was hard to follow how memory allocation of 
variables are managed and how a new code bit affects the entire program. 
In addition, \verb|xournal| is
based on GNOME canvas UI widget, which is one of soon-to-be-deprecated components 
of the GNOME project.
This led me to think about making a new note-taking program of my own from scratch. 

When I envisioned what the new "xournal" should look like, I came up with the idea of "emacs"
 of pen note-taking program. Without doubt, \verb|emacs| is one of the most
successful long-surviving text editor software. It is old but evolves continuously to adjust 
itself to the needs.
The frame/window/buffer architecture
of \verb|emacs| turned out very flexible accommodating many interesting 
ideas. Elisp, the LISP language used for \verb|emacs| scripting, is a very 
useful machinery for extending and empowering \verb|emacs|, sometimes very unexpectedly. 
Such extensibility is in part originated from the fact that 
elisp is a functional programming language. With functions as first-class
citizens, modular design is more easily achievable in functional 
programming languages. Developing a new note taking program in 
a functional programming language should have similar benefit in design and extensibility. 


In this project, I have chosen haskell for the language. Haskell is a statically typed *pure* 
functional programming language. The language is very clear, concise and elegant.
Purity, which means a function in haskell is a really a function without side effects, is 
one of the most important properties. Especially in concurrent and parallel programming paradigm, 
pure functional languages show a great potential and draw big attention in research and industry. 
The haskell community is rapidly growing and so the library space in haskell is now 
filled with increasing flow of contributions.  
Although Graphical User Interface (GUI) programming in haskell is still new and not 
very common, it turned out that we have enough useful libraries for this development.
It is also worth to mention that there is some interesting progress in haskell in seeking 
for new revolutionary directions of GUI programming, so-called {\em functional reactive programming (FRP)}. 
(However, at this moment, I do not consider switching to FRP paradigm yet.) 



Since November 2011, I started the development under the name of 
%\hyperref{http://ianwookim.org/hxournal}
{\verb|hxournal|}. The name honors 
the \verb|xournal| project. Now, after some discussion about the name in 
online community, I decided to change the name to "hoodle" where "h" comes from haskell 
and "oodle" is in a sense of doodle. The progress has been impressively fast owing to 
the productivity of the language. \verb|hoodle| has only 10k lines of codes,
which is significantly smaller than \verb|xournal| and almost 1/5 of \verb|xournalpp|
(New C++ version of \verb|xournal| from scratch). As of Dec 2012, it has a baby 
scripting feature. The scripting feature will provide a similar path to \verb|emacs| 
for community-contributed development.  

%\mainmatter 

\clearpage
\pagenumbering{arabic}
\setcounter{page}{1}

\part{Basics}


%----------------------------------------------------------------------------------------
%	CHAPTER 1
%----------------------------------------------------------------------------------------

\chapterimage{chapter_head_2.pdf} % Chapter heading image

\chapter{Introduction}

\section{What is Hoodle? Why another note-taking program?}

Hoodle is a pen-notetaking program written in haskell.


\chapter{Getting Started}

\section{Installation}
\subsection{Detect your devices}

\chapter{Basic Drawing Tool} 

\section{Understanding Hoodle Interface}

\chapter{Manipulating Selection}%

\section{Lasso selection}

\chapter{File Management}

\section{test}

%%%
%%%  Part 2 
%%%

\part{Advanced}

\chapter{Using Document Database}
\section{Hoodle d-bus server}

\chapter{Scripting Hoodle}
\section{test}

\chapter{Publishing Hoodle}
\section{test}

\chapter{Network}
\section{test}

\chapter{Recovery} 
\section{test}

%%%
%%%  Part 3
%%%

\part{Reference Manual}


%----------------------------------------------------------------------------------------
%	BIBLIOGRAPHY
%----------------------------------------------------------------------------------------

\chapter*{Bibliography}
\addcontentsline{toc}{chapter}{\textcolor{ocre}{Bibliography}}
\section*{Books}
\addcontentsline{toc}{section}{Books}
\printbibliography[heading=bibempty,type=book]
\section*{Articles}
\addcontentsline{toc}{section}{Articles}
\printbibliography[heading=bibempty,type=article]

%----------------------------------------------------------------------------------------
%	INDEX
%----------------------------------------------------------------------------------------

\cleardoublepage
\setlength{\columnsep}{0.75cm}
\addcontentsline{toc}{chapter}{\textcolor{ocre}{Index}}
\printindex

%----------------------------------------------------------------------------------------

\end{document}